% Template per generare

\documentclass[a4paper,11pt]{article}
\usepackage{lmodern}
\renewcommand*\familydefault{\sfdefault}
\usepackage{sfmath}
\usepackage[utf8]{inputenc}
\usepackage[T1]{fontenc}
\usepackage[italian]{babel}
\usepackage{indentfirst}
\usepackage{graphicx}
\usepackage{tikz}
\newcommand*\circled[1]{\tikz[baseline=(char.base)]{
    \node[shape=circle,draw,inner sep=2pt] (char) {#1};}}
\usepackage{enumitem}
% \usepackage[group-separator={\,}]{siunitx}
\usepackage[left=2cm, right=2cm, bottom=3cm]{geometry}
\frenchspacing

\newcommand{\num}[1]{#1}

% Macro varie...
\newcommand{\file}[1]{\texttt{#1}}
\renewcommand{\arraystretch}{1.3}
\newcommand{\esempio}[2]{
  \noindent\begin{minipage}{\textwidth}
    \begin{tabular}{|p{11cm}|p{5cm}|}
      \hline
      \textbf{File \file{input (da stdin)}} & \textbf{File \file{output (su stdout)}}\\
      \hline
      \tt \small #1 &
      \tt \small #2 \\
      \hline
    \end{tabular}
  \end{minipage}
}

% Dati del task
\newcommand{\gara}{.}
\newcommand{\nome}{Nel bosco c'\`e un ometto}
\newcommand{\nomebreve}{funghi}

\begin{document}
  
  
  % Intestazione
  \noindent{\Large \gara}
  \vspace{0.5cm}
  
  \noindent{\Huge \textbf \nome~(\texttt{\nomebreve})}
  \vspace{0.2cm}\\

  \begin{flushright}
      Questo problema \`e una rielaborazione del task JEDAN, COCI 2012/2013, $6^{th}$ round.
  \end{flushright}
  
  % Descrizione del task
  \section*{Descrizione del problema}
    
  \noindent
  Il bosco misterioso \`e una sequenza di $n$ funghi in fila indiana.
  La descrizione compiuta del bosco \`e costituita da $n$ numeri naturali
  $h_1, h_2, \ldots, h_n$, le altezze di detti funghi.
  Inizialmente, tutte queste $n$ altezze erano settate a $0$.
  Poi, in settembre, con le prime piogge,
  si \`e assistito ad una sequenza imprecisata di scatti di crescita.
  Ogni scatto di crescita \`e un processo che si articola nelle seguenti fasi:
  \begin{itemize}
     \item[1.] si individua un intervallo del bosco $(i,j)$ che \`e un plateau nel senso che tutti i funghi ricompresi nell'intervallo hanno la stessa altezza, cio\`e $h_k=h_i$ per ogni $k = i,i+1, \ldots, j$.
        Per come \`e fatta questa mossa, essa prevede che il plateau selezionato contenga almeno $3$ funghi, ossia $j\geq i+2$) 
     \item[2.] si incrementano di $1$ le altezze di tutti i funghi del plateau tranne il primo e l'ultimo.
   \end{itemize}

   Ad esempio, le righe della seguente tabella sono i fotoscatti che ricostruiscono la storia di un bosco.

\begin{table}[h!tb]
\begin{tabular}{ cccccc|c }
  \hline
  $h_1$ &  $h_2$ & $h_3$ & $h_4$ & $h_5$ & $h_6$ & {\sc Time}\\
  \hline
     0  &    0  &    0  &    0  &    0  &    0 &      0    \\
     0  &    1  &    1  &    1  &    1  &    0 &      1    \\
     0  &    1  &    1  &    2  &    1  &    0 &      2    \\
  \hline
\end{tabular}
\end{table}  

Quando andiamo ad esplorare il bosco, ad esempio quello che corrisponde all'ultima riga della tabella qui sopra, non \`e detto che riusciamo a coglierne ogni singolo aspetto.
Potremmo ad esempio leggere

\begin{table}[h!tb]  
\begin{tabular}{ cccccc|c }
  \hline
  $h_1$ & $h_2$ & $h_3$ & $h_4$ & $h_5$ & $h_6$ & {\sc Time}\\
  \hline
    -1  &  -1  &   -1  &    2  &   -1  &   -1 &      1    \\
  \hline
\end{tabular}
\end{table}  

dove i valori $-1$ rappresentano delle letture mancanti.\\

Il nostro problema \`e il seguente:
data una lettura di un bosco vorremmo determinare quante sono le possibili realt\`a di bosco compatibili con la lettura ottenuta.

\section*{Dati di input}
  
Questa volta input ed output non avvengono tramite file.
Come input leggete due righe da stdin.

La prima riga contiene l'intero positivo $n$ ($1\leq n \leq 10\,000$),
il numero di funghi nel bosco.

La successiva riga contiene gli $n$ interi $h_i$, $i=1,\ldots, n$,
separati da uno spazio; essi rappresentano le letture ottenute sulle altezze dei funghi. Quando $h_i\neq -1$ esso esprime precisamente l'altezza del fungo $i$-esimo.


\section*{Dati di output}
  Come output, dovete stampare su stdout una sola riga
contenente un intero non negativo: il numero di possibili boschi compatibili con la lettura ricevuta modulo $1\, 000\, 000\, 007$.

  
  \section*{Subtask}
  Un totale di $100$ punti \`e ripartito sui seguenti subtask.
  \begin{itemize}
    \item \textbf{Subtask 1 [0 punti]:} i casi di esempio.
    \item \textbf{Subtask 2 [10 punti]:} $n\leq 15$.
    \item \textbf{Subtask 3 [10 punti]:} $n\leq 50$.
    \item \textbf{Subtask 4 [10 punti]:} $n\leq 50$.
    \item \textbf{Subtask 5 [10 punti]:} $n\leq 100$.
    \item gli altri subtask sono tutti da $2$ o $3$ punti ciascuno.
  \end{itemize}

% Esempi
\section*{Esempio di input/output}
\setlength{\tabcolsep}{6pt}
\esempio{
3

-1 2 -1
}{0}
  
\esempio{
3

-1 -1 -1
}{2}
  
\esempio{
6

-1 -1 -1 2 -1 -1
}{3}
  

\end{document}
